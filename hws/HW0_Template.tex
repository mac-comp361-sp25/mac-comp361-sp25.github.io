\documentclass{exam}
\usepackage{graphicx} % Required for inserting images
\usepackage{listings}
\usepackage{amsmath}
\usepackage{algorithmicx}
\usepackage{algpseudocode}
\usepackage{geometry}[border=1in]
\usepackage{algorithm}
\usepackage{amsmath}
\usepackage{amssymb}
\usepackage{amsthm}
\usepackage{listings}
\usepackage{mathtools}

\newtheorem*{statement}{Statement}

\title{HW0}
\author{COMP361 --- Suhas Arehalli}
\date{Spring 2025}

\begin{document}
\maketitle

\begin{questions}
    \question \textit{(Sipser 0.4)} If $A$ has $a$ elements and $B$ has $b$ elements, how many elements are in $A \times B$? Explain your answer

    \question \textit{(Sipser 0.5)} If $C$ is a set with $c$ elements, how many elements are in the power set of $C$, $\mathcal{P}(C)$? Explain your answer. 

    \question \textit{(Sipser 0.11)} Let $S(n) = 1 + 2 + 3 \dots + n$ be the sum of the first $n$ natural numbers and let $C(n) = 1^3 + 2^3 + \dots + n^3$ be the sum of the first $n$ cubes. Prove the following equalities to show that $C(n) = S^2(n)$ for every $n \geq 1$. 

    \begin{parts}
        \part $S(n) = \frac{1}{2} n(n+1), \forall n \geq 1$
        \part $C(n) = \frac{1}{4} n^2(n+1)^2, \forall n \geq 1$
    \end{parts}

    \question \textbf{\textit{(*Sipser 0.12)}} Find the error in the following proof that all horses are the same color.

    \noindent\fbox{\begin{minipage}{\linewidth}
     \begin{statement}
         In a set of $h$ horses, all horses are the same color. 
     \end{statement}
     
    \begin{proof}
         By induction on $h$.
         
        \textbf{Base Case:} For h = 1. In any set containing just one horse, all horses are clearly the same color.
        
        \textbf{Inductive Step:} For $k \geq 1$, assume the the claim is true for $h=k$ and prove that it is true for $h = k+1$. 

        Take any set $H$ of $k+1$ horses. We show that all the horses in this set are the same color. Remove one horse from this set to get the set $H_1$. By the induction hypothesis, all the horses in $H_1$ are the same color. Now replace the removed horse and remove a different horse to obtain the set $H_2$. By the same argument, all horses in $H_2$ are the same color. Therefore, all of the horses in $H$ must be the same color, and the proof is complete.
    \end{proof}
    \end{minipage}}
        
    \question \textbf{\textit{(*Sipser 0.13)}} Show that every graph with 2 or more nodes contains two nodes that have equal degrees.

\end{questions}

\end{document}

